%----------------------------------------------------------------------------------------
%	Metropolia Thesis LaTeX Template
%----------------------------------------------------------------------------------------
% License:
% This work is licensed under the Creative Commons Attribution 4.0 International License.
% To view a copy of this license, visit http://creativecommons.org/licenses/by/4.0/.
%
% However, this license apply to this template. As a template, it is supposed to be
% modified for your own needs (with your thesis content). For this reason, if you use
% this project as a template and not specifically distribute it as part of a another
% package/program, we grant the extra permission to freely copy and modify these files as
% you see fit and even to delete this copyright notice.
% In short, you are free to publish your thesis under whatever license you wish, even
% keep the all rights reserved to you.
%
% Authors:
% Panu Leppäniemi, Patrik Luoto, Mikaa Oni and Patrick Ausderau
%
% Credits:
% Panu Leppäniemi: abstract, def, cleaning,...
% Patrik Luoto: title page, abstract in Finnish, abbreviation, math,...
% Mikaa Oni: switch to biber biblatex
% Patrick Ausderau: initial version, style, table of content, bibliography, figure,
%                   appendix, table, source code listing,...
%
% Please:
% If you find mistakes, improve this template and alike, please contribute by sharing
% your improvements and/or send us your feedback there:
% https://github.com/panunu/metropolia-thesis-latex
% And of course, if you improve it, add yourself as an author.
%
% Compiler:
% Use XeLaTeX as a compiler. LuaLaTeX works too.
% Typical compilation:
% # minted require -shell-escape to run  external script.
% # -8bit avoid ^^I for tabs in minted.
% $ xelatex -shell-escape -8bit main
% # If any change in the bibliography
% $ biber main
% # If any change with the abbreviation or acronym
% $ makeglossaries main
% #Then compile again
% $ xelatex -shell-escape -8bit main
% #And if still some citation or label warnings, compile once more
% $ xelatex -shell-escape -8bit main

%----------------------------------------------------------------------------------------
%	THESIS INFO
%----------------------------------------------------------------------------------------

% All general information (main language, title, author (you), degree programme, major
% option, etc.)
% Edit the file chapters/0info.tex to change these information
\documentclass[12pt,a4paper,oneside,article]{memoir}%Do not touch this first line ;)

% Global information (title of your thesis, your name, degree programme, major, etc.)

\def\bilingual{yes}%For Finnish students, you must have 2 abstracts, one in English and one in your native language (Finnish or Swedish), so "yes", your thesis is bilingual.
%\def\bilingual{no}%For international student writing in English, only one language and one abstract.

\def\thesislang{english} %change this depending on the main language of the thesis.
%\def\thesislang{english} % "english" is the only other supported language currently. If someone has the swedish, please contribute!

\def\secondlang{finnish} %if the main language is Finnish (or Swedish), you must have 2 abstracts (one in Finnish (or Swedish) and one in English)
%\def\secondlang{finnish}
%If the main language is English and that you are native Finnish (or Swedish) speaker, you must have also abstract in your native language on top of the English one.

\author{Jere Luhtanen} %your first name and last name

%\def\alaotsikko{Alaotsikko/Subtitle} %DISABLED, seems not to be an option with the 2018 template. If you really need it, uncomment and modify style/title.tex accordingly.

%License
%When publishing your thesis to theseus.fi, you can keep all rights reserved to you,
%or use one of the Creative Commons https://creativecommons.org/licenses/?lang=en
%This attribute will set the license in the metadata of the generated pdf.
%possible options (case sensitive):
%all (keep all rights reserved to yourself)
%by (Attribution)
%by-sa (Attribution-ShareAlike)
%by-nd (Attribution-NoDerivs)
%by-nc (Attribution-NonCommercial)
%by-nc-sa (Attribution-NonCommercial-ShareAlike)
%by-nc-nd (Attribution-NonCommercial-NoDerivs)
%Note that theseus do not accept dedication to public domain CC0
\def\thesiscopy{all}

%Finnish section, for title/abstract
\def\otsikko{Dynamic Ring Buffer}
\def\tutkinto{Insinööri (AMK)} % change to your needs, e.g. "YAMK", etc.
\def\kohjelma{Tieto- ja viestintätekniikka}
\def\suuntautumis{Smart Systems}
\def\thesisfi{Insinöörityö}%was Opinnäytetyö
\def\ohjaajat{
Keijo Länsikunnas\newline
}
\def\tiivistelma{
Insinöörityön aiheena on C++ kielellä ohjelmoitu dynaaminen rinkibufferi. Työn tavoitteena oli kehittää standardikirjaton säiliöiden vaatimuksia vastaava automaattisesti kokoaan muuttava bufferi joka tarjoaaa FIFO ja LIFO ominaisuuksia. Työ toteutettiin yhtiön Rightware Oy pyynnöstä iteratiivisen suunnittelutieteen menetelmällä. Työn lopputuloksena oli toimiva ohjelmisto, testikirjasto sekä tehokkuusmittaustulokset.  \newline

Tämä on tiivistlemän toinen kappale.
}
\def\avainsanat{Ohjelmointi, C++, dynaaminen säiliö}
\def\aihe{Dynaamisen rinkipuskurin kehittäminen C++ ohjlemointikielellä}%for the pdf metadata/properties. If not used, empty it and also the \def\subject.

%English section, for title/abstract
\title{Dynamic ring buffer}
\def\metropoliadegree{Bachelor of Engineering} % change to your needs, e.g. "master", etc.
\def\metropoliadegreeprogramme{Information Technology}
\def\metropoliaspecialisation{Smart Systems}
\def\thesisen{Bachelor’s Thesis} % change to your need, e.g. master's
\def\metropoliainstructors{
Keijo Länsikunnas, Project Manager\newline
}
\def\abstract{
This is a report for a development work of a dynamic ring buffer in the C++ programming language . The goal of this work was to develop a container that satisfies the standard library requirements and provides FIFO and LIFO features, along with dynamic size allocation. This work was proposed by a company Rightware Oy via design science methodology. The outcome was a functioning software library, a test suite and performance benchmarks comparing against existing solutions.\newline

beginning of second paragraph\ldots
}
\def\metropoliakeywords{Developement, C++, dynamic container}
\def\subject{Development of a dynamic ring buffer in the C++ programming language}%for the pdf metadata/properties. If not used, empty it and also the \def\aihe.

%----------------------------------------------------------------------------------------
%	GLOBAL STYLES
%----------------------------------------------------------------------------------------

% If you need extra package, etc. modify the style/style.tex file.
% If you are using Windows OS, you will need to change default font to Arial in that
% style/style.tex file (or install Liberation Sans font to your system).
% If you are using MacOS or linux, make sure you have Liberation Sans font installed.
\input{style/style.tex}
% Normally, you do not need to modify the title style. It's content comes from the
% chapters/0info.tex file.
\input{style/title.tex}

%----------------------------------------------------------------------------------------
%	ABBREVIATION AND GLOSSARY
%----------------------------------------------------------------------------------------

% Add/edit all your acronyms, abbreviations, glossary entries, etc. definitions in
% chapters/0abbr.tex file.
% You can have as many as you wish. Only the ones you use in your text (inserted with
% \gls{} command) will print in the Glossary/Lyhenteet.
% Generate the glossary
\makeglossaries

% Acronyms, abbreviations, etc.

\newacronym{stl}{STL}{Standard Template Library}
\newacronym{hmi}{HMI}{Human Machine Interface}
\newacronym{lrai}{LRAI}{Legacy Random Access Iterator}
\newacronym{ui}{UI}{User Interface}
\newacronym{fifo}{FIFO} {First-In-First-Out}
\newacronym{lifo}{LIFO} {Last-In-First-Out}
\newacronym{dsm}{DSM} {Design Science Methodology}

% Glossary entries

\newglossaryentry{part_key}{
	name={partition key},
	description={a column or set of columns from the same table whose consolidated value decide the partition for a given data}
}
\newglossaryentry{thesis}{
	name=thesis,
	description={a written essay one submitted for a university degree},
	plural=theses
}
\newglossaryentry{latex}
{
	name=\LaTeX{},
	description={Is a mark up language specially suited for scientific documents}
}

\newglossaryentry{maths}
{
	name=mathematics,
	description={Mathematics is what mathematicians do}
}




%----------------------------------------------------------------------------------------
%	DOCUMENT STARTS HERE...
%----------------------------------------------------------------------------------------

\begin{document}
\IfLanguageName{finnish}{
}{
  \raggedright%2021 template, align left, no hyphennization for English version
}
\counterwithout{listing}{chapter}

%----------------------------------------------------------------------------------------
%	TITLE PAGE
%----------------------------------------------------------------------------------------

\input{style/title_headers.tex}
\maketitle
\newpage

%----------------------------------------------------------------------------------------
%	ABSTRACT / Tiivistelmä
%----------------------------------------------------------------------------------------

% If you are international student writing in English, ignore the Finnish abstract.
% If you are Finnish citizen, you must have 2 abstracts, one in Finnish (or Swedish
% depending on your mother tongue) and one in English regardless of the main language of
% your thesis. Normally, you do not need to modify the abstract style. It's content comes
% from the chapters/0info.tex file.
\ifdefstring{\bilingual}{no}{%
    \input{style/abstract_en.tex}
    }{%
    \IfLanguageName{finnish}{%order of abstracts based on main language and spacing hell
        \input{style/abstract_fi_fi.tex}
        \input{style/abstract_fi_en.tex}
        }{
        \input{style/abstract_en.tex}
        \input{style/abstract_fi.tex}
    }
}
%----------------------------------------------------------------------------------------
%	License? Acknowledgement?
%----------------------------------------------------------------------------------------

% Uncomment next line and edit chapters/0license.tex if you want license in your thesis.
%\input{chapters/0license.tex}

% Uncomment next line and edit chapters/0acknowledgement.tex if you want acknowledgements.
%\input{chapters/0acknowledgement.tex}

%----------------------------------------------------------------------------------------
%	TABLE OF CONTENTS
%----------------------------------------------------------------------------------------

\input{style/toc.tex}

%list of figure, tables would come here if relevant?

%----------------------------------------------------------------------------------------
%	Lyhenteet / Abbreviation
%----------------------------------------------------------------------------------------

% If you don't use abbreviations/glossary, remove the following line.
% Abbreviation and Glossary
% Normally, you don't have to modify this file. Your abbreviations, etc. goes in
% ../chapters/0abbr.tex file.

\begin{singlespacing}

	% \gsladdall%would add all terms even if not used in your text.
	\addtocontents{toc}{\cftpagenumbersoff{chapter}}
	{
		\vspace{-.46cm}
		\titleformat{\section}
		{\fontsize{13.5pt}{13.5pt}\bfseries\normalfont}
		{\thesection}{.5cm}{}
		\renewcommand*{\glossarypreamble}{\vspace{.7\baselineskip}}
		%Adapt labelwidth (sorry for the ugly hack)
		\setlist[description]{leftmargin=!, labelwidth=4em}
		\IfLanguageName {finnish} {
			\printacronyms[title=Lyhenteet]
			}{
			\printacronyms[title=List of Abbreviations]
		}
		\setlist[description]{leftmargin=!, labelwidth=7em}
		\vspace{1cm}
		\printglossary
		\setlist[description]{style=standard} % reset settings back to default
	}
	\addtocontents{toc}{\cftpagenumberson{chapter}}
\end{singlespacing}

\clearpage


%----------------------------------------------------------------------------------------
%	CONTENT
%----------------------------------------------------------------------------------------

\input{style/content.tex}%reset page number to 1, etc.

% Thesis content if you strictly follow the "Final Year Project guide". Of course, you
% can adapt to your specific needs (add more chapter, rename them, etc.).
% Introduction

\chapter{Introduction}

As technology integrates more thoroughly in to our everyday lives, efficient data storage and and handling is in a crucial role for any high performance application. In C++ programming language the \gls{stl} provides efficient and robust solutions to store and access data. However, the diverse landscape of software systems sometimes poses unique problems that require specialized data structures.

The goal of this project was to provide a specialized tool for handling data streams, one that is more flexible and efficient compared to existing solutions. This work describes the design and implementation of a dynamic ring buffer in relation to other \gls{stl} containers. This container does not only adhere to the case company's requirements but satisfies the requirements set by the \gls{stl} for containers and additionally combines both \gls{fifo} and \gls{lifo} features within a single structure.  The primary difference from pre-existing solutions is the ability to dynamically change its capacity to accommodate fluctuating data volumes.

The case company specialises in automotive \gls{hmi} development tools, with their market leading product that offers both a design and editing tool and a runtime environment for automotive \gls{ui}. The dynamic ring buffer developed in this thesis is meant to provide a capable and flexible solution for varying data loads while maintaining fast basic operations. Handling varying high volume data streams efficiently is mandatory in an environment where performance and memory is limited.

The beginning of this thesis is dedicated to exploring the existing solutions and landscape of the problem, which is followed by the used technologies and implementation details about the container itself. At the end this will all be summarized alongside pure benchmark data comparing the solution to \gls{stl} containers and providing conclusive notes.
% uncomment what you need.
% Project Specifications
%\clearpage%if the chapter heading starts close to bottom of the page, force a line break and remove the vertical vspace
\vspace{21.5pt}
\chapter{Project Specifications}

The programming space is like a pyramid that relies on existing layers upon layers for developers to utilise and build amazing things without having to start from scratch or worry about implementation details <[TODO]insert reference of interface programming>. Therefore this field is filled with tradition, expectations and unspoken rules that are implicitly inherited when coding or studying. At the same time, much like all things on the internet, it is very chaotic and the user is always responsible for verifying the validity of the product they are consuming.

This project tries aims to act as much as possible like a \gls{stl} container so that developers that are familiar with the C++ programming language could use it with ease and know what it does without referencing the documentation. To achieve this goal the project adheres to two sets of requirements.

-  The \gls{stl} container requirements and all its sub-requirements

-  The requirements set by the case company.

Although some of the individual points of both sets overlap, they are different by purpose. The \gls{stl} requirements specify what a container is and how it can be used, and the company requirements tilts more into performance and details of implementation.

\subsection{Case company requirements}

The case company has required the product the fulfill these requirements 

\begin{table}[h]
  \centering
  \caption{Requirements by the case company}%IMPORTANT the caption must be before the tabular, so it will be on top of the table (there are other tricks to force it on top; but this one is straightforward).
  \vspace{-16.5pt}%time to time, spacing between caption and table can go too big...
  \begin{tabular}{| 
  >{\centering\arraybackslash}p{.20\textwidth} | 
  >{\centering\arraybackslash}p{.20\textwidth} | 
  >{\centering\arraybackslash}p{.20\textwidth} | 
  >{\centering\arraybackslash}p{.20\textwidth} |}
    \hline
    Requirement & Description & Rationale & Validation  \\
    \hline
    C++14 Compliance &  The ring buffer must be implemented using C++14 standards. & Ensures compatibility with C++ features and environments. & Code review and compilation tests \\
    \hline
    \gls{stl} Compatibility & Must satisfy \gls{stl} named container requirements. & Allows integration with other \gls{stl} containers
    and algorithms & Test suite \\
    \hline 
    Class template & Container must be a class template. & Allows the container to hold any data type. & Test suite \\
    \hline 
    Container adapter support & Must satisfy requirements imposed by \gls{stl} queue, stack and priority\_queue & Allows adapter integration into the container. & Test suite \\
    \hline
    Constant time operations &  Support constant time move, swap and access operations. & Critical for performance in real-time systems. & Performance benchmarks and profiling. \\
    \hline
    Bidirectional iteration & Must support constant and non-constant iteration forwards and backwards. & Required for general algorithms and usage.[TODO] & Test suite \\
    \hline
    \gls{lrai} support & Iterators must satisfy \gls{lrai} requirements. & Required for iterative algorithms & Test suite \\
    \hline
    Amortized constant time insertion and removal & Must support amortized constant time insertion and removal at both ends of the buffer & Performance benchmakrs and profiling.\\
    \hline
    
  \end{tabular}
  \label{table:some_data}
\end{table}

The ring buffer shall be written in C++14. \\
The ring buffer shall be an STL compatible container.\\
The ring buffer shall be a template class, taking the contained element type as a template parameter.\\
The ring buffer shall satisfy the requirements imposed by the container adaptors std::queue, std::stack, and std::priority\_queue.\\
The ring buffer shall support constant time move construction, move assignment, and swapping.\\
The ring buffer shall support constant and non-constant iteration (logically) forwards and backwards.\\
The ring buffer iterators shall satisfy the requirements of LegacyRandomAccessIterator.\\
The member function size() shall return the number of elements in the ring buffer in constant time.\\
The member function empty() shall return whether the ring buffer lacks elements in constant time.\\
The member function data() shall return a pointer to the logically and physically first element of a physically contiguous sequence \\containing all the elements of the ring buffer in linear time.
The ring buffer shall support amortized constant time insertion and removal of elements at both logical ends of the buffer.\\
The ring buffer shall never overwrite an existing element.\\
The ring buffer shall support constant time element access with and without bounds checking.\\
The ring buffer shall not initialize nor in any way manipulate element slots that reside logically outside the ring buffer.\\
Each member variable, function, parameter, return value, precondition, postcondition, side effect, exception, and exception safety \\guarantee shall be documented.
The ring buffer shall be deployable in commercial software without restrictions.\\
Each testable requirement shall have a test.\\
The ring buffer shall be benchmarked for time and memory usage in differing use cases against suitable implementations of std::vector, \\std::deque, and std::list, where applicable.

%Bridge to second chapter
Although some of the individual points of both sets overlap, they are different by purpose. The requirements set a clear goal for the project, but as in programming in general, there are multiple ways of reaching the goal and not all roads are made equal. 


These requirements not only act as testable milestones for the development of the project, but also as a reference for learning via the design science methodology.
%\input{chapters/methods.tex}
%\input{chapters/theory.tex}
%\input{chapters/solution.tex}
%\input{chapters/conclusion.tex}

% Sample content to demonstrate LaTeX command. You will likely delete this line and the
% next \input{sample/*} lines. You are also safe to delete the sample/ folder and its
% content once you refershed your LaTeX skills. Also check the appendix samples.
%sample content to demonstrate LaTeX command.
\vspace{21.5pt}
\chapter{Demonstration Content}\label{demo:content}

This is a chapter to demonstrate some of the \gls{latex} commands that you can use to format your text. If you are new to it, \gls{latex} follows the \gls{wysiwym} idea similar to \gls{html} where you concentrate on the content and structure and leave the formatting and styling to the computer. You will write your content in a plain text file\footnote{which can not get corrupted and can be put under version control} and indicate the structure with commands (similar to \gls{html} tags) then compile it to generate the pdf. Check some books or tutorials such as \LaTeX{} Wikibooks\footnote{\url{https://en.wikibooks.org/wiki/LaTeX}} and try with an online editor such as Overleaf\footnote{\url{https://www.overleaf.com/}} so you do not need to install the compiler on your computer first.

\section{Text, terms and abbreviations, figures, lists, etc.}

In \gls{latex}, the \mintinline{tex}{\textbf{bold}} command produces \textbf{bold}, \mintinline{tex}{\textit{italic}}  \textit{italic} and nesting \mintinline{tex}{\textbf{\textit{bolditalic}}} generates \textbf{\textit{bolditalic}}. If the goal is to semantically mark importance, then use \emph{emphasize} with \mintinline{tex}{\emph{emphasize}} command. That would take care of cases such as \mintinline{tex}{\textit{text in italic with \emph{important stuff} inside}} which will compile to \textit{text in italic with \emph{important stuff} inside}. Note that a paragraph is added by forcing a new line.

When one want to use an abbreviation or acronym like \gls{ddeah} using the \mintinline{tex}{\gls{ddeah}} command in \gls{latex}, the first time, it comes with it full version as can be seen in first paragraph of chapter \ref{demo:content} and for all next usages in its short form. Of course, when needed, the full version is available using e.g. the \mintinline{tex}{\acrlong{someID}} command. The defined terms like \gls{maths} use the same \mintinline{tex}{\gls{math}} \gls{latex} command. The Capitalized can be obtained with \mintinline{tex}{\Gls{id}}.

In this template, the abbreviations are defined in the \texttt{chapters/0abbr.tex} file with the \mintinline{tex}{\newacronym{id}{SHORT}{Long Form}} command. There can be many abbreviations and terms defined there, only the ones that are used in the text will be printed in the list of abbreviations (after table of content). And of course, it is the job of the compiler to sort them alphabetically. Should be avoided; but to have all the abbreviations and terms, even the non-used ones, use the \mintinline{tex}{\glsaddall} command before printing the list of abbreviations.

\begin{itemize}
  \item Check the thesis guide about the orphan list item (\mintinline{tex}{\item}): if only first item in the page, force a new page \mintinline{tex}{\clearpage} before \mintinline{tex}{\begin{itemize}}.
  \item When the list items are not sentences, they begin with a lowercase letter, and the last list item ends in a period.
  \item When the list items are sentences, they begin with a capitalized letter, and the list items end in a period. If item of the list contains a long text that spans multiple lines, the left edge aligns automatically.
\end{itemize}

And let also try the figure (see figure \ref{fig:latex-cover}) and internal reference (with \mintinline{tex}{\label{lbl:id}} and \mintinline{tex}{\ref{lbl:id}}). Alternative text is obtained with custom made \mintinline{tex}{\AltText{text}} command (using pdfcomment and accsupp packages). The reference can be done to any label, for example why not to appendix \ref{appx:first} or to appendix \ref{appx:second}? To note, \gls{latex} will place the figure to the best place (except with forcing). Let them float till the final of final edit\ldots ~then force them to not break a paragraph.%hugly hack... I'm sorry

\begin{figure}[ht]
  \centering
  \AltText{meaningful alternative description (e.g. LaTeX, from typeseting to genrated pdf)}{\includegraphics[width=7.1cm]{LaTeX_cover}}
  \caption{\gls{latex} cover image (Copied from \citeauthor{wikibooks:latex} (\citedate{wikibooks:latex}) \cite{wikibooks:latex}).}
  \label{fig:latex-cover}
\end{figure}

According to the thesis guide, there must be a paragraph of text between figures/tables/listings and a chapter/(sub)section. And a paragraph is many sentences long like at least two.

\section{Bibliography references and citations}

Here is an example how to cite a bibliography entry \cite{kopka:guide} using the \\\mintinline{tex}{\cite{kopka:guide}} \gls{latex} command \cite[section 4.1]{tobias:book}. You might also like \mintinline{tex}{\citeauthor{some:id}} and \mintinline{tex}{\citedate{some:id}} ~as demonstrated in figure \ref{fig:latex-cover} caption and others like \mintinline{tex}{\citetitle{some:id}}, \mintinline{tex}{\citeurl{some:id}}, etc. To get very precise references, like chapter, section, pages number or range of a book, timing in video,\ldots that get indicated in square brackets in the command like \mintinline{tex}{\cite[04:01]{youtube:biblatex}} \cite[04:01]{youtube:biblatex}. Check the thesis guide, if the reference is only for the current sentence, the \mintinline{tex}{\cite{}} is placed before the dot, if the reference is for entire paragraph or group of sentences, then after the dot. If there is multiple sources for one sentence or paragraph, they have to be grouped together using the \mintinline{tex}{\cites[pp. 3, 5, 9--13]{some:id}[chap 4]{other:id}{more:id}{etc:id}} command. \cites[pp. 216--220]{kopka:guide}[chapter Special Pages, sections 3--4]{wikibooks:latex}[section 4.1]{tobias:book}[04:01--04:19]{youtube:biblatex}

Like for abbreviations, the bibliography entries are stored in a separated \texttt{biblio.bib} file and only the \mintinline{tex}{\cite}d ones will be printed in the bibliography references. There are many entry types such as \mintinline{tex}{@book}, \mintinline{tex}{@article}, \mintinline{tex}{@online}, \mintinline{tex}{@video}, \mintinline{tex}{@thesis} and many more, e.g. see \cite[section 2.1]{ctan:biblatex}. In the worst case, there is the \mintinline{tex}{@misc} fallback entry type. \gls{latex} and biber compilers will take care of the numbering and sorting of the cited entries. Some tools help in managing the entries such as OttoBib\footnote{\url{http://www.ottobib.com/}} that will generate book entry from \gls{isbn} or ZoteroBib\footnote{\url{https://zbib.org/}} that can also take \gls{url}, \gls{doi}, etc. It is also possible to get the entry form the IEEE Xplore\footnote{\url{https://ieeexplore.ieee.org/}} or Google Scholar\footnote{\url{https://scholar.google.com/}} as shown in figure \ref{fig:bibtex}. Of course, even if using such tools can greatly help, manual check/edit might be required (e.g. missing author,\ldots).

\begin{figure}[ht]
  \centering
  \AltText{getting BibTeX entries from Google Scholar and IEEE Xplore}{\includegraphics[width=\textwidth]{bibtex_gscholar_ieeexplore}}
  \caption{BibTeX entries from Google Scholar (left) or IEEE Xplore (right)}
  \label{fig:bibtex}
\end{figure}

Let's also try a long quote from the \citetitle{un:udhr}
\begin{quote}
(1) Everyone has the right to education. Education shall be free, at least in the elementary and fundamental stages. Elementary education shall be compulsory. Technical and professional education shall be made generally available and higher education shall be equally accessible to all on the basis of merit.

(2) Education shall be directed to the full development of the human personality and to the strengthening of respect for human rights and fundamental freedoms. It shall promote understanding, tolerance and friendship among all nations, racial or religious groups, and shall further the activities of the United Nations for the maintenance of peace.

(3) Parents have a prior right to choose the kind of education that shall be given to their children. \cite[article 26]{un:udhr}
\end{quote}

%TODO example with cite interview/conversation (unpublished or misc)
%https://tex.stackexchange.com/questions/111363/exclude-fullcite-citation-from-bibliography
\textit{Quisque augue} est, \textbf{elementum ac porttitor} non, porttitor ac orci. Donec hendrerit, ligula ac luctus egestas, sem dolor pretium nunc, sed vehicula magna diam a massa. Donec mattis, arcu et tempor mattis, risus tortor ultrices metus, nec sodales sem dolor eu elit.

Nullam egestas enim at odio pellentesque bibendum.

\subsection{Subsection}
Donec et sapien ac leo condimentum vulputate id et tellus. Maecenas hendrerit malesuada interdum. Aenean dignissim sem faucibus elit congue faucibus id non risus. Morbi at dui non tortor pellentesque consequat non eget urna. Cras in sapien dui, a tincidunt velit.
\reaction{\label{eq:reaktio}$\underset{\text{+II}}{\ce{2Fe^2+}}$ + $\underset{\text{+I\;-I}}{\ce{H2O2}}$ + $\underset{\text{+I\;-II}}{\ce{2H3O^+}}$ <=> $\underset{\text{+III}}{\ce{2Fe^3+}}$ + $\underset{\text{+I\;-II}}{\ce{4H2O}}$}
Työn aluksi rauta(II)ionit hapetetaan rauta(III)ioneiksi väkevällä vetyperoksidilla, kuten reaktion~\ref{eq:reaktio} hapetusluvuista nähdään (rauta hapettuu, happi pelkistyy).
\reaction{Fe^3+( \emph{aq} ) + 3OH^-( \emph{aq} ) + $(x-1)$H2O( \emph{l} ) -> FeOOH $\cdot$ $x$(H2O)( \emph{s} )}
Rauta(III)ionit saostetaan emäksen (\ce{NH3}) avulla ja saadaan tuotteeksi kidevedellinen rauta(III)hydroksidi. Saatu saostuma pestään \ce{NH4NO3}:lla.
\reaction{FeOOH $\cdot$ $x$(H2O)( \emph{s} ) ->T[$\Delta$900-1000\celsius] Fe2O3( \emph{s} )}

\subsection{Subsection with \texorpdfstring{\Gls{maths}}{Mathematics}}%gls inside chapter/section/... will generate hyperref warning (link inside link in table of content), to avoid that, use the \texorpdfstring
Donec et sapien ac leo condimentum vulputate id et tellus. Maecenas hendrerit malesuada interdum. Aenean dignissim sem faucibus elit congue faucibus id non risus. Morbi at dui non tortor pellentesque consequat non eget urna. Cras in sapien dui, a tincidunt velit. Tertiäärinen butyylikloridi reagoi veden kanssa oheisen reaktion mukaisesti:
\reaction{(CH3)3CCl + 2H2O -> (CH3)3COH+H3O+ +Cl-}
Kyseessä on ensimmäisen kertaluvun reaktio, joten reaktion nopeus on
\begin{align}
v=-\frac{\mathrm{d}[\tn{t-ButCl}]}{\mathrm{d}t}=\frac{\mathrm{d}[\tn{HCl}]}{\mathrm{d}t}=k[\tn{t-ButCl}]
\end{align}
Jos tarkastellaan lähtöaineen t-butyylikloridin häviämistä saadaan
\begin{align}
\frac{\mathrm{d}[\tn{t-ButCl}]}{[\tn{t-ButCl}]}&=-k\mathrm{d}t \\
\int \frac{\mathrm{d}[\tn{t-ButCl}]}{[\tn{t-ButCl}]}&=-k \int \mathrm{d}t \\
\ln \int_{[\tn{t-ButCl}]_0}^{[\tn{t-ButCl}]} [\tn{t-ButCl}]&=-k\int_0^t t \\
\ln \left( \frac{[\tn{t-ButCl}]}{[\tn{t-ButCl}]_0} \right)&=-kt
\end{align}
Ionivahvuus lasketaan kaavalla.
\begin{align}
I&=\frac{1}{2}\cdot\sum z_i^2c_i \\
z_i&= \tn{ionin varausluku} \nonumber \\
c_i&= \tn{ionin konsentraatio} \nonumber
\end{align}
Aktiivisuuskerroin $\gamma_\pm$ lasketaan kaavalla.
\begin{align}
\log \gamma_\pm &= -\left|z_+\cdot z_-\right|A\cdot I^{\frac{1}{2}} \\
A &= \tn{0,509 (lämpötilassa 25\celsius}) \nonumber \\
I &= \tn{ionivahvuus} \nonumber \\
z &= \tn{ionien varaus} \nonumber
\end{align}

\section{Section with Source Code}
Donec et sapien ac leo condimentum vulputate id et tellus. Maecenas hendrerit malesuada interdum. Aenean dignissim sem faucibus elit congue faucibus id non risus. Morbi at dui non tortor pellentesque consequat non eget urna. Cras in sapien dui, a tincidunt velit.

%For sharelatex users, use space instead of tab to avoid ^^I
\begin{code}
  \begin{minted}{php}
<?php
$username = $_POST["username"];
//maybe not?
if ($userName){
  ?>
  <h2>Hello <?= $username ?>!</h2>
  <p>your message got received.</p>
  <?php
}
?>
\end{minted}
\captionof{listing}{Descriptive Caption Text (e.g. this \gls{php} code do blah)}
  \label{code:testphp}
\end{code}


As see in listing \ref{code:testphp}, blah. It is also possible to have code inline, for example \mintinline{sql}{SELECT * FROM user WHERE age >= 18} that was \gls{sql}.
The lisings \ref{code:htmlfull} and \ref{code:htmlpart} show how to load code from an external source file. In the case of listing \ref{code:htmlpart} it only take few line out of the source code file.

\begin{code}
  \inputminted{html}{code/html5_sample.html}
  \captionof{listing}{Some \gls{html} code}
  \label{code:htmlfull}
\end{code}
 Donec et sapien ac leo condimentum vulputate id et tellus. Maecenas hendrerit malesuada interdum. Aenean dignissim sem faucibus elit congue faucibus id non risus.

 \begin{code}
   \inputminted[firstline=3,lastline=6]{html}{code/html5_sample.html}
   \captionof{listing}{The \mintinline{html}{<head>} section of an \gls{html} page}
  \label{code:htmlpart}
\end{code}


 Morbi at dui non tortor pellentesque consequat non eget urna. Cras in sapien dui, a tincidunt velit.


\section{Section with Table}
Donec et sapien ac leo condimentum vulputate id et tellus. Maecenas hendrerit malesuada interdum. Aenean dignissim sem faucibus elit congue faucibus id non risus. Morbi at dui non tortor pellentesque consequat non eget urna. Cras in sapien dui, a tincidunt velit.

\begin{table}[h]
  \centering
  \caption{Some data}%IMPORTANT the caption must be before the tabular, so it will be on top of the table (there are other tricks to force it on top; but this one is straightforward).
  \vspace{-16.5pt}%time to time, spacing between caption and table can go too big...
  \begin{tabular}{| l | >{\centering\arraybackslash}p{.5\textwidth} |}
    \hline
    Test 1 & test 1234 test \\
    \hline
    Some more data comes here & with more values and if the text is very long it will disappear out of the box unless you force the column size :( You can use e.g. \textbackslash raggedright or \textbackslash centering (as in this example) to avoid hyphenization of long words\ldots \\
    \hline
  \end{tabular}
  \label{table:some_data}
\end{table}

As presented in table \ref{table:some_data}: Donec et sapien ac leo condimentum vulputate id et tellus. Maecenas hendrerit malesuada interdum. Aenean dignissim sem faucibus elit congue faucibus id non risus. Morbi at dui non tortor pellentesque consequat non eget urna. Cras in sapien dui, a tincidunt velit.

\begin{table}[h]
  \centering
  \caption{Another table with tabularx}
  \begin{tabularx}{.95\textwidth}{| l | >{\centering\arraybackslash} X |}
    \hline
    Test 1 & test 1234 test \\
    \hline
    Some more data comes here & with more values and if the text is very long it will disappear out of the box unless you force the table size :( \\
    \hline
  \end{tabularx}
  \label{table:some_data2}
\end{table}

As presented in table \ref{table:some_data2}: Donec et sapien ac leo condimentum vulputate id et tellus. Maecenas hendrerit malesuada interdum. Aenean dignissim sem faucibus elit congue faucibus id non risus. Morbi at dui non tortor pellentesque consequat non eget urna. Cras in sapien dui, a tincidunt velit.

Donec et sapien ac leo condimentum vulputate id et tellus. Maecenas hendrerit malesuada interdum. Aenean dignissim sem faucibus elit congue faucibus id non risus. Morbi at dui non tortor pellentesque consequat non eget urna. Cras in sapien dui, a tincidunt velit.

\begin{table}[htbp]
  \centering
  \caption{Booktabs example}
  \vspace{-16.5pt}
    \begin{tabular}{rrrr}
    \toprule
    t (s) & [HCl] & [t-ButCl] & $\ln\frac{[t-ButCl]}{[t-ButCl]_0}$ \\
    \midrule
    0     & 4,02  & 160,88 & 0,00 \\
    10    & 63    & 101,9 & -0,46 \\
    20    & 115,2 & 49,7  & -1,17 \\
    30    & 141,3 & 23,6  & -1,92 \\
    40    & 157,9 & 7     & -3,13 \\
    50    & 161   & 3,9   & -3,72 \\
    60    & 164,3 & 0,6   & -5,59 \\
    70    & 163,5 & 1,4   & -4,74 \\
    80    & 163,8 & 1,1   & -4,99 \\
    90    & 164,1 & 0,8   & -5,30 \\
    100   & 164,3 & 0,6   & -5,59 \\
    \bottomrule
    \end{tabular}
  \label{tab:thisislabel}
\end{table}

As presented in table \ref{tab:thisislabel}: Donec et sapien ac leo condimentum vulputate id et tellus. Maecenas hendrerit malesuada interdum. Aenean dignissim sem faucibus elit congue faucibus id non risus. Morbi at dui non tortor pellentesque consequat non eget urna. Cras in sapien dui, a tincidunt velit.

\input{sample/2lorem.tex}
\input{sample/3graph.tex}

%----------------------------------------------------------------------------------------
%	BIBLIOGRAPHY REFERENCES
%----------------------------------------------------------------------------------------

\input{style/biblio.tex}

%----------------------------------------------------------------------------------------
%	APPENDICES
%----------------------------------------------------------------------------------------

\input{style/appendix.tex}
%force smaller vertical spacing in table of content
%!!! There can be some fun depending if the appendices have (sub)sections or not :D
% You will have to play with these numbers and eventually add the \vspace line  before
% some \chapter and force another number.
% To add more fun, time to time the table of content get wrong after a build :(
\addtocontents{toc}{\vspace{11pt}}
\pretocmd{\chapter}{\addtocontents{toc}{\protect\vspace{-24pt}}}{}{}

\liite{1}% This is a hack to have right page numbering for each appendix. Make sure to
% use a unique number for each appendix.
\input{sample/Xappendix1.tex}% Sample content to demonstrate appendix in LaTeX. You
% are safe to delete this lines (and the next samples) once you refreshed your LaTeX
% skills (and safe to delete the sample folder and all its file too).

%\addtocontents{toc}{\vspace{11pt}}%fix vertical space for Table of Content
\liite{2}
\input{sample/Xappendix2.tex}

\addtocontents{toc}{\vspace{11pt}}
\liite{3}
\input{sample/X_R_example.tex}


%----------------------------------------------------------------------------------------
%	THIS IS THE END
%----------------------------------------------------------------------------------------
\end{document}
