\documentclass[12pt,a4paper,oneside,article]{memoir}%Do not touch this first line ;)

% Global information (title of your thesis, your name, degree programme, major, etc.)

\def\bilingual{yes}%For Finnish students, you must have 2 abstracts, one in English and one in your native language (Finnish or Swedish), so "yes", your thesis is bilingual.
%\def\bilingual{no}%For international student writing in English, only one language and one abstract.

\def\thesislang{english} %change this depending on the main language of the thesis.
%\def\thesislang{english} % "english" is the only other supported language currently. If someone has the swedish, please contribute!

\def\secondlang{finnish} %if the main language is Finnish (or Swedish), you must have 2 abstracts (one in Finnish (or Swedish) and one in English)
%\def\secondlang{finnish}
%If the main language is English and that you are native Finnish (or Swedish) speaker, you must have also abstract in your native language on top of the English one.

\author{Jere Luhtanen} %your first name and last name

%\def\alaotsikko{Alaotsikko/Subtitle} %DISABLED, seems not to be an option with the 2018 template. If you really need it, uncomment and modify style/title.tex accordingly.

%License
%When publishing your thesis to theseus.fi, you can keep all rights reserved to you,
%or use one of the Creative Commons https://creativecommons.org/licenses/?lang=en
%This attribute will set the license in the metadata of the generated pdf.
%possible options (case sensitive):
%all (keep all rights reserved to yourself)
%by (Attribution)
%by-sa (Attribution-ShareAlike)
%by-nd (Attribution-NoDerivs)
%by-nc (Attribution-NonCommercial)
%by-nc-sa (Attribution-NonCommercial-ShareAlike)
%by-nc-nd (Attribution-NonCommercial-NoDerivs)
%Note that theseus do not accept dedication to public domain CC0
\def\thesiscopy{all}

%Finnish section, for title/abstract
\def\otsikko{Dynamic Ring Buffer}
\def\tutkinto{Insinööri (AMK)} % change to your needs, e.g. "YAMK", etc.
\def\kohjelma{Tieto- ja viestintätekniikka}
\def\suuntautumis{Smart Systems}
\def\thesisfi{Insinöörityö}%was Opinnäytetyö
\def\ohjaajat{
Keijo Länsikunnas\newline
}
\def\tiivistelma{
Insinöörityön aiheena on C++ kielellä ohjelmoitu dynaaminen rinkibufferi. Työn tavoitteena oli kehittää standardikirjaton säiliöiden vaatimuksia vastaava automaattisesti kokoaan muuttava bufferi joka tarjoaaa FIFO ja LIFO ominaisuuksia. Työ toteutettiin yhtiön Rightware Oy pyynnöstä iteratiivisen suunnittelutieteen menetelmällä. Työn lopputuloksena oli toimiva ohjelmisto, testikirjasto sekä tehokkuusmittaustulokset.  \newline

Tämä on tiivistlemän toinen kappale.
}
\def\avainsanat{Ohjelmointi, C++, dynaaminen säiliö}
\def\aihe{Dynaamisen rinkipuskurin kehittäminen C++ ohjlemointikielellä}%for the pdf metadata/properties. If not used, empty it and also the \def\subject.

%English section, for title/abstract
\title{Dynamic ring buffer}
\def\metropoliadegree{Bachelor of Engineering} % change to your needs, e.g. "master", etc.
\def\metropoliadegreeprogramme{Information Technology}
\def\metropoliaspecialisation{Smart Systems}
\def\thesisen{Bachelor’s Thesis} % change to your need, e.g. master's
\def\metropoliainstructors{
Keijo Länsikunnas, Project Manager\newline
}
\def\abstract{
This is a report for a development work of a dynamic ring buffer in the C++ programming language . The goal of this work was to develop a container that satisfies the standard library requirements and provides FIFO and LIFO features, along with dynamic size allocation. This work was proposed by a company Rightware Oy via design science methodology. The outcome was a functioning software library, a test suite and performance benchmarks comparing against existing solutions.\newline

beginning of second paragraph\ldots
}
\def\metropoliakeywords{Developement, C++, dynamic container}
\def\subject{Development of a dynamic ring buffer in the C++ programming language}%for the pdf metadata/properties. If not used, empty it and also the \def\aihe.